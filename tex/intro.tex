\section{Introduction}
\pagenumbering{arabic}
Introductory sentences.

The rapid urbanisation of planet Earth is presenting both challenges and opportunitites in the way that finite resources are used in light of the challenges of climate change and resource depletion. \ldots This project therefore seeks to firstly take a broad look at the global picture of urbanisation and the challenges it presents, the policy background which arises out of this state of affairs and the ways in which the opportunities offer solutions to the challenges. Having considered the `big picture', the project focuses in specifically on the development of mathematical models  which optimise the integrated provision and management of energy, water and waste resources in an urban area. This Early Stage Assessment report therefore has two main aspects which reflect the ground to be covered in this study. Firstly, the big picture is considered as global urban trends are outlined and their implications as challenges and opportunities. The `urban metabolism' concept is introduced as an underlying theoretical concept which can be used to describe the links between trends, challenges and opportunities. This background information then motivates a research question. Secondly, the report moves on to describe various optimisation models in existence in the fields of energy, water and waste which can be used as a starting point in the development of an integrated resource management model, before going on to show the first steps already taken in developing a model.

The subject of this thesis is an interesting topic of study. It requires both a broad understanding of the big picture of urban resource management in three areas (energy, water and waste), including knowledge of policy environments etc. coupled with an understanding of mathematical programming for optimsation models.

The final outcome of this study will be a suite of tools which can be used to optimise the provision of water, waste and energy resources within an urban area, at a minimal financial and environmental cost, subject to various constraints. Before the methodology and models are described, the context of the project is outliend in this introduction, to motivate their development. Firstly, global urban trends are summarised before outlining the challenges and opportunities that come with urban areas. The concept of `urban metabolism' is introduced, which will provide the underlying theoretical framework for the study. Features of urban resource flows are then discussed in order to highlight opportunities for `resource integration'. Having established the context of the study, a precise `research question' is formulated to guide the rest of the research.

Who is this useful for: urban planners etc.

\subsection{The world is urbanising}
Three things can be said about urbanisation globally: (1) it is on the \emph{increase}; (2) this fact is \emph{important}; and (3) it follows from (1) and (2) that urbanisation is \emph{innevitable}. Firstly, the \emph{increase} in the global urban population is considered.

Currently, urban areas are home to about half of the global population \citep{AREAS2012}\footnote{Note: statistics in this report will generally be from 2011 or earlier. More recent figures will not be recorded because they will need to be updated for the final submission anyway.}. By 2030, it is expected that the urban population will rise to 61\% of the total, and again to 70\% by 2050, as cities absorb population growth, and continue to receive migrants from rural areas, aand rural areas become reclassified as urban \citep{Cohen2006}. Two notable features of this urbanisation are that firstly, most of the growth is concentrated in the developing world, with Africa and Asia together accounting for 86\% of the increase in urban dwellers by 2050 \citep{York2011}; and that despite there being much attention on `megacities', the majority of urban growth will be in smaller cities (below 500000 people) in the forseable future). Cities of more than 10 million residents will accomodate less than 10\% of the urban population \citep{Cohen2006}.

These forecasts aren't just the view of a small number, but widely accepted opinion. Thus even \citet{Cohen2004} who takes a skeptical look at urbanisation forecasts (pointing out that historical forecasts can be overestimates by more than 75\%; and that there is no universally consistent definition of `urban'\footnote{See for example \citet{DepartmentforCommunitiesandLocalGovernment2006}.} concludes: \emph{``Neverthelss, despite all the problems of error and inaccuracy and the long-standing definitional problems that have never been overcome, it is clear that the world is still in the midst of a sweeping and profound urban transformation that is literally changing the face of the planet.''}

It is natural to ask why such trends in urbanisation are seen, and whether this is good or bad. A useful description of the mechanism of urbanisation has been given by \citet{WorldBank2008a}. Urban centres arise as people migrate from the countryside for employment and better access to services (such as education, healthcare and employment). As cities grow, they beging to specialise with firms taking advantages of economies of scale. Cities become `agglomeration economies' when multiple types of firms and industries occupy the space, facilitating knowledge spillovers, and serving each other, such that individual firms can do their job better. At the same time, transport (and hence trade) costs reduce. Thus scale economies and agglomeration economies act to increase trade, and hence the prosperity of an area, attracting more migrants to the area. Thus a virtuous circle arises whereby city size and economic growth feed each other. This principle is seen in practise, given that one source estimates that 80\% of global GDP is generated in urban areas \citep{AREAS2012}.

This isn't to say that cities are without problems: the widespread presence of slums being one obvious concern (the World Bank describes these as `growing pains', and notes that they were historically seen in cities like London, but are no longer in existence there). However, in general, urbanisation is a clear route to economic prosperity. It is good news that the urbanisation patterns seen historically in OEDC countries are being replicated in the developing world \citep{WorldBank2008a}. This leads to the conclusion that urbanisation in \emph{important}.

Finally then, it must be the case that urbanisation is \emph{innevitable} due to the economic prosperity it brings to the developing world. Thus it is troubling to learn that \emph{``\ldots 72\% of developing countries have adopted policies designed to stem the tide of migration to their cities.''} GUARDIAN REFERENCE.

\subsection{Urbanisation poses challenges}
Having considered urbanisation trends and their benefits, the challenges of urbanisation are now considered. There are two primary challenges---those of \emph{environmental strain} and \emph{economic sustainability}.

Considering the environmental strain, there are multiple problems associated with urban areas including land-use impacts, resource scarcity, poor water quality, emissions, the heat-island effect and eutrophicaion \citep{Cai2011}. A society starts to over-consume resources when the area (hinterlands) which serve it can't provide enough food, water and and other materials to meeds the needs of the inhabitants. A cycle ensues when an increasing population is required to managee the land to extract resources for that population. Thus even agriculturial driven societies (as well as industrial driven societies) can face resource scarcity \citep{Haberl2001a, Haberl2001b, Gr2003}. Even now individual cities are able to overcome the pressures on resources by looking beyond the immediate hinterlands, and making use of the transport system to be dependent on transport links, this is just spreading and growing the global environmental impact cities are responsible for \citep{Agudelo-Vera2011}. Thus, whilst currently home to just over 50\% of the global population, urban areas are responsible for around 60\% of primary energy use and 71\% of the global energy-related greenhouse gas emissions \citep{IEA2008}. For example, \citet{Grubler2009} shows the relationship between urban sprawl and the emissions from the associated increase in transport use (as journeys are required to be more frequent and longer).

Urbanisation results in a growing requirement for urban water delivery in view of growth in the urban population and a demand for a higher quality of life resulting in `water stress'. Associated with the increasing demand is the need to treat the water and prevent contamination. This demand for water in high quantitiies comes with a financial burden \citep{Diagger2009}. \citet{Kennedy} illusrates the problem of urban water sustainability by considering the relationship between urbanisation and groundwater. Shallow wells are sufficient to supply a population in the urban development stage, with wastewater discharging to a watercourse. As the population grows and water extraction rates rise, the water table falls requiring the digging of deeper wells and any shallow groundwater can become contminated by discharge wastewater. Thus water must be sourced (at significant expense) from outside the city, changing the city's groundwater characteristics such that the water table can rise above its original level and even cause flooding. In summary then, water security issues in urban areas are causing economic, health and environmental concerns.

Moving to consider waste management, there are a number of issues that urban areas face. For example, the risk of contaminating groundwater and public opposition to waste disposal near residential areas \citep{Li2006}, as well as growing waste generation rates and decreasing disposal capacity \citep{Lu2009}. In Greece, for example, municipal solid waste management has reached a `critical point'.

Moreover, the challenges faced in the energy, water and waste field are interalated. Population growth increases demands for water and energy significantly, and economic growth increases this demand on a per capita basis as people improve their living standards. With increased water demands, there will come increasing energy demands (for treatment and distribution) and increased wastewater generation. Furthermore, climate change will limit the availability of water in many places and will therefore induce high levels of spending and energy consumption in extracting water resources for human use (through desalination and other treatment methods). The increase in energy consumption will then exacerbate climate change invoking a self-reinforcing challenge \citet{Webber2011}.

There is now an increasing emphasis on developing policies to tackle the issues thrown-up by urbanisation. For example, \citet{Agudelo-Vera2011} argues that the objective of urban planning should include sustainable development (pointing out that urban planning has always sought to solve problems, such is accomodating larger populations and meeting transport requirements). It then points out that resource management is an essential part of sustainable development, and that resource management shouldn't remain in the past where it's main aim was to meet demand. Thus, resource management needs to be integrated into urban planning in order to end the pattern of over consumption and excessive waste production which the global eco-system cannot carry. A specific policy example is the European Union target of 20\% of final energy to be provided by renewable sources by 2020 \citep{Keirstead2012}.

London provides a good example of a city experiencing the problems and therefore developing policies described in this section. For example, London has a comprehensive range of policies concerning transport, the retrofitting of housing and the decentralisation of energy supply in order to reduce energy-related emissions \citep{Davis2011}. In the water sector, London is under great stress, suffering from an old network which leaks about 25\% of the water which enters it. When this is coupled with an increasing supply (as population grows, with a tendency to live in houses of fewer people) and the reduced availability of water due to climate change, then it is obvious that current consumption (already above the national average) is unsustainable. Thus policies are being implemented to capture rainwater and wastewater as a resource; change consumer behaviour and payment patterns; and make buildings more efficient \citep{Nickson2011}. Finally, considering London's municipal waste, 49\% currently ends up in landfill, some of this outside London; policies are being implemented to build new waste management infrastructure, increase recylcling and use waste incineration as a method of energy production, for example with the SELCHP incinerator \citep{Zabal2011}.
%\citep{Chen2006} Link to resource use, environment, policy and technology options.

%\citep{Batt2010} to summarise.
In summary, this study is in agreement with \citet{Newman1999} that the sustainability of cities needs to be improved through better resource management. It must be remembered however, that a city isn't simply a resource proceeing machine, but it is the home to many people, and the place where they seek economic opportunity.  Therefore, there is a particular challenge to better manage resources without harming the `livability' of cities. Another conclusion that is innevitable from the survey of challanges posed by cities is that resource efficiency isn't enough. The world is already struggling to sustain the global population, so a `per capita' decrease in consumption in a growing population will not be enough on it's own (for example, at current trends, the entire global biomass would be required to meet energy needs by 2050). There must be dematerialisation in absolute terms, especially when a more efficient use of resources may cause people to seek a higher standard of living---the so called `take-back' effect \citep{Winiwater2011}.

% Economics
\citep{Bettencourt2007} Innovation cycles


However, as well as presenting these challenges, there are also unique opportunities inherent to urban spaces which can be exploited. These will be explored in the next section.

\subsection{Opportunities afforded by urban areas}
The density of urban areas, and the resulting co-location of infrastructures is the source of a potential solution to the challenges posed by urban areas.

\citep{Agudelo-Vera2011} History of changing purposes/strategies in urban design. 
'There is a need for a comprehensive framework that integrates RM and UP.'
Rees2009 quote about 'fully integrating'.

An underlying theoretical framework is required which can help us measure the scale of the challenges presented; the opportunities on offer; and the success of the models. The 'urban metabolism' concept will serve this purpose.

Many studies suggest integrating infrastructures within their own field. This project is just an extension of that concept.

The definition of integration (literature may be different from this study).

Furthermore, there is little research in this field (despite many calls for it).  

\subsection{Urban metabolism---the theoretical framework}
Urban metabolism is the theoretical concept which supports this study in three ways:
\begin{enumerate}
	\item Urban metabolism studies give data which motivates the development of the models.
	\item It suggests am objective to achieve in the model: namely, a move away from resource `linearisation' towards `circularisation'.
	\item It provides a way of measuring the success of the models.
\end{enumerate}
Can be used to quantify the `challenges' discussed in that section.

\subsection{Aims and scope of the study}
Having now outlined the context and motivation for the study, a `research question' is posed. This gives the study a clear focus, and provides a way of testing the success of the final outcome of the study.
The following question will be answered:
\begin{quote}
	\emph{By how much can the metabolism of an urban area be improved by creating and implementing models which optimise the integrated provision of energy, water and waste?}
\end{quote}
This question comprises three aims:
\begin{enumerate}
	\item Research the motivation, opportunity and methods for such models for the literature review. This will bring together up-to-date information on urban trends; a review of urban resource interactions and infrastrucures; and knowledge from existing methods used in optimising energy, water and waste management.
	\item Develop models to calculate the optimal transfer of resources through a network of processes such that demand for resources of required quality is met. The models must capture the complexity of considering multiple resources and their qualities whilst remaining tractable. Three models will be developed during the term of study:
		\begin{enumerate}[(i)]
			\item A prototype model based on a small subsistence-based community. There will be no spatial or temporal disaggregation in this model, because its primary purpose is to develop a methodology which can handle multiple resource types, each of which are associated with one or more quality parameters.
			\item A model to optimise resource management in an urban development (likely to be part of a city in China). This will introduce spatial disaggregation into the model.
			\item A water-based model to optimise the management of the energy-water nexus. Unlike the previous two models (which are focused on design), this model will consider system operation, and will therefore include temporal disaggregation.
		\end{enumerate}
	\item Assess how well the models improve urban metabolism through highlighting technological opportunities and pathways, and as a consequence, how well they minimise environmental impacts and ensure economic stability for urban areas.
\end{enumerate}

There are four (potentially) novel aspects to this research: 
\begin{itemize}
	\item Whilst there are tools which consider the optimisation of two resources (for example, the minimisation of energy requirements for meeting water demands in a town), there are no such tools which simultaneously optimise energy, water and waste management. 
	\item Secondly, a new method will be required which incorporates resource quality requirements in a manner that makes the model computationally efficient and tractable. 
	\item Thirdly, the application of optimisation methods to the field of urban metabolism is new ground, with the literature dominated by accounting studies.
	\item Fourthly, the tools developed will be new to an audience of urban planners, policy makers and utility service companies.
\end{itemize}



\subsection{Report structure}
Having presented the background to the study to provide motivation for the research question posed, the report continues by reviewing existing methods of optimising the provision of energy, water and waste in Section~\ref{sec:methods}. In Section~\ref{sec:models}, prototype models at their initial stage are presented. The report conludes in Section~\ref{sec:conc} with preliminary results, and an outline of the work to be undertaken in the rest of the project.

In summary, it has been argued cities need to focus on resource integration in order to reduce damage to the environment and economic risk. Currently, there are no models which consider the optimal management of multiple urban resources.

\subsection{Summary}
In conclusion, there is strong case to be made that research into a model that optimises the provision of urban energy, water and waste resources should be persued. This is because urbanisation has socioeconomic advantages, such that it should not be resisted, despite the challenges it brings. Rather, urban centres provide an opportunity with the proximity and co-location of infrastructure to meet the challenges of resource management. On top of this there is a research-gap in the field of integrated resource provision.
