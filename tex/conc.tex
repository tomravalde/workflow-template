\section{Conclusions and further work}
\label{sec:conc}

\subsection{Conclusions}
Global urban trends have been examined and it has been found that the world is rapidly urbanising---especially in developing countries. Furthermore, it has been suggested that this urbanisation brings economic opportunity to the residents of towns and cities, thus any approach that seeks to mitigate against the environmental and economic challanges of urban areas must do so without opposing urbanisation itself. It is widely viewed that improved resource efficiency through integrated resource management is the best way to go about tackling these problems. Despite common agreement here, there is surprisingly little in the way of tools for optimising the integrated provision of energy, water and waste resources. A review of optimised provision models within these fields has been undertaken with a view to developing an integrated resource model in this study. The first steps in developing a model have optimised resource flows in a small village (albeit with significant simplifications from reality). The first steps in developing a model which can incorporate resource quality concerns has been created.

\subsection{Further work}
The following tasks are suggested for further work:
\begin{itemize}
	\item The resource quality model for Tat needs to be expanded to consider the whole water network in Tat.
	\item Once a working water network has been developed, it can be integrated into the Tat model, along with quality considerations for other resources, such as biomass. 
	\item Following this, more sophisticated resource integration models can be developed which take into account temporal and spatial disaggregation.
\end{itemize}

\subsubsection*{Thesis outline}
The following is suggested as a chapter outline for the final submission.
\begin{enumerate}
	\item Introduction
	\item The urban context: trends, challenges and opportunities
	\item Urban metabolism: a measure of resource circularisation
	\item Methodological background: a review of existing modelling methods in the energy, water and waste sectors
	\item Development of an integrated resource management model
	\item A prototype model for a village
	\item A model for urban design with spatial dissagregation
	\item A model for operation, with spatial and temporal disaggregation
	\item Conclusions and further work
\end{enumerate}

