\begin{abstract}
The globe is rapidly undergoing urbanisation, such that by 2050, it is estimated that 70\% of the world's population will live in urban areas. Urban areas are important because of the economic and social benefits they bring to their inhabitants, yet they have significant challenges associated with them. Urban areas a reach a point at which they can no longer be sustained by their immediate environment, and thus look to wider markets for the maintenance and growth of their population; contributing to environmental damage, such as resource scarcity and global warming. The over consumption of resources also has economic repercussions, such that as resources become limited, the area needs to become more innovative in order to postpone economic stagnation and collapse. Whilst being hubs of resource over consumptions, cities are also home to the oppoutunities to manage resources more efficiently and achieve overall dematerialisation. This is because of the numerous infrastucture interconnections, synergies and integration opportunities made possible by the co-location of infrastructure within the urban environments---especially between technologies and management processes in the energy, water and waste sectors---for example in hydropower generation and energy from waste incineration. The concept of urban metabolism can be used to measure the ecological sensetivity of a city and evaluate its success at moving from resource linearisation to resource circularisation.

Despite numerous models in individual disciplines which optimise the design and operation of systems that manage energy, water and waste, there is very little in the literature which takes a broad view and considers how the provision of all three resource might be optimised, taking integration opportunities into consideration. The first steps of the development of such a model are taken in this study, adopting a generic superstructure approach to optimse the flow of resources through various processes in a small agricultural based village, which is solved using mixed integer linear programming to meet demand at minimum financial and environmental cost. The next step in model development is to take into account the quality (not just the quantity) of resources to be allocated, and expanded to consider larger areas and spatial and temporal disaggregation.
\end{abstract}
