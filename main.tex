\documentclass[a4paper]{article}\usepackage{graphicx, color}
%% maxwidth is the original width if it is less than linewidth
%% otherwise use linewidth (to make sure the graphics do not exceed the margin)
\makeatletter
\def\maxwidth{ %
  \ifdim\Gin@nat@width>\linewidth
    \linewidth
  \else
    \Gin@nat@width
  \fi
}
\makeatother

\IfFileExists{upquote.sty}{\usepackage{upquote}}{}
\definecolor{fgcolor}{rgb}{0.2, 0.2, 0.2}
\newcommand{\hlnumber}[1]{\textcolor[rgb]{0,0,0}{#1}}%
\newcommand{\hlfunctioncall}[1]{\textcolor[rgb]{0.501960784313725,0,0.329411764705882}{\textbf{#1}}}%
\newcommand{\hlstring}[1]{\textcolor[rgb]{0.6,0.6,1}{#1}}%
\newcommand{\hlkeyword}[1]{\textcolor[rgb]{0,0,0}{\textbf{#1}}}%
\newcommand{\hlargument}[1]{\textcolor[rgb]{0.690196078431373,0.250980392156863,0.0196078431372549}{#1}}%
\newcommand{\hlcomment}[1]{\textcolor[rgb]{0.180392156862745,0.6,0.341176470588235}{#1}}%
\newcommand{\hlroxygencomment}[1]{\textcolor[rgb]{0.43921568627451,0.47843137254902,0.701960784313725}{#1}}%
\newcommand{\hlformalargs}[1]{\textcolor[rgb]{0.690196078431373,0.250980392156863,0.0196078431372549}{#1}}%
\newcommand{\hleqformalargs}[1]{\textcolor[rgb]{0.690196078431373,0.250980392156863,0.0196078431372549}{#1}}%
\newcommand{\hlassignement}[1]{\textcolor[rgb]{0,0,0}{\textbf{#1}}}%
\newcommand{\hlpackage}[1]{\textcolor[rgb]{0.588235294117647,0.709803921568627,0.145098039215686}{#1}}%
\newcommand{\hlslot}[1]{\textit{#1}}%
\newcommand{\hlsymbol}[1]{\textcolor[rgb]{0,0,0}{#1}}%
\newcommand{\hlprompt}[1]{\textcolor[rgb]{0.2,0.2,0.2}{#1}}%

\usepackage{framed}
\makeatletter
\newenvironment{kframe}{%
 \def\at@end@of@kframe{}%
 \ifinner\ifhmode%
  \def\at@end@of@kframe{\end{minipage}}%
  \begin{minipage}{\columnwidth}%
 \fi\fi%
 \def\FrameCommand##1{\hskip\@totalleftmargin \hskip-\fboxsep
 \colorbox{shadecolor}{##1}\hskip-\fboxsep
     % There is no \\@totalrightmargin, so:
     \hskip-\linewidth \hskip-\@totalleftmargin \hskip\columnwidth}%
 \MakeFramed {\advance\hsize-\width
   \@totalleftmargin\z@ \linewidth\hsize
   \@setminipage}}%
 {\par\unskip\endMakeFramed%
 \at@end@of@kframe}
\makeatother

\definecolor{shadecolor}{rgb}{.97, .97, .97}
\definecolor{messagecolor}{rgb}{0, 0, 0}
\definecolor{warningcolor}{rgb}{1, 0, 1}
\definecolor{errorcolor}{rgb}{1, 0, 0}
\newenvironment{knitrout}{}{} % an empty environment to be redefined in TeX

\usepackage{alltt}
\usepackage{healy-tufte-style}

\immediate\write18{sh ./vc}
\input{vc} % Uncomment if not using version control

\usepackage{fontspec}
\setromanfont[]{Linux Libertine}
\setsansfont[]{Linux Biolinum}
\setmonofont[]{Liberation Mono}

\title{Modelling approach comparison study}
\author{Tom Ravalde\footnote{\texttt{thomas.ravalde08@imperial.ac.uk}} \\[0.3em]
\emph{Civil and Environmental Engineering}}
\date{\today}

\begin{document}
\maketitle
\thispagestyle{empty}

\bigskip \bigskip
\begin{abstract}
	The city of the future needs to reduce their intake and output of resources in order to minimise environmental damage. One way to achieve this is to improve the urban metabolism of such areas. This means that resources (such as energy, water and waste) are reused in processes as much as possible. Currently however, there exist no modelling approaches to optimise systems which deal with such distinct resources. Here, three possible methods are proposed and evaluated with a view to answering the question `how can an urban area most effectively improve its metabolism?'
\end{abstract}

\section{Introduction}

\subsection{Null model}
This is the most basic model built. Some text with a footnote\footnote{This is a footnote}.

Objective :
\begin{align}
\min_{I_r,E_r,F_p} \Bigg\{&\sum_{r}^R c_rI_{r} \Bigg\} \\
\end{align}
Quantity balance (mass and energy)
\begin{align}
I_r+G_r &= D_r + E_r \\
\end{align}

Resource production
\begin{align}
G_r&=\sum_{p}^{P}k_{pr} F_p \\
\end{align}

Emissions limit
\begin{align}
\sum_{p}^P e_p F_p &\leq \mbox{emissions target} 
\end{align}

Non-negativity constraints (alternatively, these can be specified between boundaries).
\begin{align}
I_r &\geq 0 \label{eq:tat_basic_imports} \\
E_r &\geq 0 \\
F_p &\geq 0 
\end{align}

\subsection{PRaQ model}
PRaQ stands for 'processes, resources and qualities'. 

Objective:
\begin{align}
\min_{I_r, F_p, G^{(res)}_{prq}, G^{(qual)}_{prq}, E_r} \Bigg\{ &\sum_q^Q \sum_r^R c_{rq} X_{rq} I_r \Bigg\} \\ 
\end{align}

Quantity balance
\begin{align}
I_r + \sum_p^P G^{(res)}_{pr} &= D^{(res)}_r + E_r \\ 
\end{align}

Quality balance
\begin{align}
\sum_r^R \Bigg( X_{rq} I_r + \sum_p^P G^{(qual)}_{prq} \Bigg) &= \sum_r^R \Bigg( D^{(qual)}_{rq} + X_{rq} E_r \Bigg) \\ 
\end{align}

Quality production
\begin{align}
G^{(qual)}_{prq} &= F_p k_{prq} \\ 
\end{align}

Balance qualities produced in each process with overall quality produced by the system. This is applied to the production/generation of a resource $r$ with a particular quality $q$ from a given process $p$, which are defined by $\delta_{prq}=1$.
\begin{align}
G^{(res)}_{pr} X_{rq} &= G^{(qual)}_{prq} \delta_{rq}, \qquad \mbox{if } \delta_{prq} = 1 
\end{align}

Non-negativity constraints (altenatively, these can be specified between boundaries)
\begin{align}
I_r &\geq 0 \\ 
E_r &\geq 0 \\ 
F_p &\geq 0 
\end{align}

\begin{knitrout}
\definecolor{shadecolor}{rgb}{0.969, 0.969, 0.969}\color{fgcolor}\begin{kframe}
\begin{alltt}
1 + 1
\end{alltt}
\begin{verbatim}
## [1] 2
\end{verbatim}
\begin{alltt}
\hlfunctioncall{plot}(\hlfunctioncall{rnorm}(100))
\end{alltt}
\end{kframe}
\includegraphics[width=\maxwidth]{figure/foo} 

\end{knitrout}


\begin{knitrout}
\definecolor{shadecolor}{rgb}{0.969, 0.969, 0.969}\color{fgcolor}\begin{kframe}
\begin{alltt}
1 + 1
\end{alltt}
\begin{verbatim}
## [1] 2
\end{verbatim}
\begin{alltt}
\hlfunctioncall{plot}(\hlfunctioncall{rnorm}(30))
\end{alltt}
\end{kframe}
\includegraphics[width=\maxwidth]{figure/bar} 

\end{knitrout}


\begin{knitrout}
\definecolor{shadecolor}{rgb}{0.969, 0.969, 0.969}\color{fgcolor}\begin{kframe}
\begin{alltt}
\hlfunctioncall{plot}(\hlfunctioncall{rnorm}(10))
\end{alltt}
\end{kframe}
\includegraphics[width=\maxwidth]{figure/blob} 

\end{knitrout}



%\begin{titlepage}


% Upper part of the page
\includegraphics[width=0.5\textwidth]{./img/ImperialLogo.jpg}\\[1cm]

\begin{center}

\textsc{\large Department of Civil and Environmental Engineering}\\[1.5cm]

\textsc{\Large A comparison of modellng approaches}\\[0.5cm]

%Title
\HRule \\[0.4cm]
{ \LARGE \bfseries Highly Integrated Urban Water, Waste and Energy Resources}\\[0.4cm]

\HRule \\[1.5cm]

% Authors and supervisor
\begin{minipage}{0.5\textwidth}
\begin{flushleft} \large
\emph{Author:}\\
Thomas \textsc{Ravalde}\\
00558022 \\
\end{flushleft}
\end{minipage}
\begin{minipage}{0.4\textwidth}
\begin{flushright} \large
\emph{Supervisor:} \\
Dr J. \textsc{Keirstead} \\
\emph{Co-supervisor:} \\
Dr I. \textsc{Stoianov} \\
\end{flushright}
\end{minipage}

\vfill

% Bottom of page
{\today}

\end{center}

\end{titlepage}

%\pagenumbering{roman} % this is returned to 'arabic' within intro.tex
%\input{./tex/abstract}
%\tableofcontents
%\listoffigures
%\listoftables

%\section{Introduction}
\pagenumbering{arabic}
There is a great need to improve the metabolism of urban areas in order to reduce inputs and outputs to the urban environment, and hence minimise environmental impacts associated with resource consumption (including resource depletion and global warming). This end requires that resources are used more efficiently, taking advantages of the opportunities that urban areas provide to integrate processes through the reuse of resources consumed and produced in various processes.

Currently though, there is a lack of methods which seek to optimise the management of multiple resource types (energy, water and waste) and their processes---this is despite the existence of numerous models within the individual disciplines. (TABULATE?)

The aim of this study therefore is to develop a model which considers multiple resource types (energy, water and waste) and their qualities and associated processes. This report describes three different approaches to the development of such a model:

TABULATE MODEL AND ITS FEATURES.

The rapid urbanisation of the globe is presenting both challenges and opportunities in the way that the Earth's finite resources are used in view of the problems presented by resource depletion and climate change. This project seeks firstly to take a broad look at global urbanisation trends and the environmental and economic challenges presented with respect to resource management; before looking at opportunities in the urban environment for more efficient use of resources through resource integration. `Urban metabolism' is introduced as the underlying theoretical concept which can be used to assess how efficiently an area uses resources. Having considered this `big picture', the project focuses specifically on the development of mathematical models which optimise the integrated provision and management of energy, water and waste resources in an urban area. Existing models in these fields are reviewed, before the first steps in developing an integrated resource model are described. It is anticipated that the final outcome of the study will be a suite of tools that work at various scales, with different degrees of temporal and spatial resolution. This will be useful to urban planners, policy makers and utility service companies. First, global trends in urbanisation are considered.
%Firstly, the big picture is considered as global urban trends are outlined and their implications as challenges and opportunities. The `urban metabolism' concept is introduced as an underlying theoretical concept which can be used to describe the links between trends, challenges and opportunities. This background information then motivates a research question. Secondly, the report moves on to describe various optimisation models in existence in the fields of energy, water and waste which can be used as a starting point in the development of an integrated resource management model, before going on to show the first steps already taken in developing a model.
%The subject of this thesis is an interesting topic of study. It requires both a broad understanding of the big picture of urban resource management in three areas (energy, water and waste), including knowledge of policy environments etc. coupled with an understanding of mathematical programming for optimsation models.
%The final outcome of this study will be a suite of tools which can be used to optimise the provision of water, waste and energy resources within an urban area, at a minimal financial and environmental cost, subject to various constraints. Before the methodology and models are described, the context of the project is outliend in this introduction, to motivate their development. Firstly, global urban trends are summarised before outlining the challenges and opportunities that come with urban areas. The concept of `urban metabolism' is introduced, which will provide the underlying theoretical framework for the study. Features of urban resource flows are then discussed in order to highlight opportunities for `resource integration'. Having established the context of the study, a precise `research question' is formulated to guide the rest of the research.

\subsection{The world is urbanising}
Three things can be said about urbanisation globally: (1) it is on the \emph{increase}; (2) urban centres are economically \emph{important}; and (3) it follows from (1) and (2) that the continued trend of urbanisation is \emph{inevitable}. 
\begin{enumerate}
	\item Firstly, the \emph{increase} in the world's urban population is considered. Currently, urban areas are home to about half of the global population \citep{AREAS2012}\footnote{Note: statistics in this report will generally be from 2011 or earlier. More recent figures will not be recorded because they will need to be updated for the final submission anyway.}. By 2030, it is expected that the urban population will rise to 61\% of the total, and to 70\% by 2050 (Figure~\ref{fig:urban_trends})\footnote{A breakdown of UN population statistics is available from \texttt{http://esa.un.org/wup2009/unup/index.asp?panel=1}.}. There are three explanations for this \citep{Cohen2006}:
	\begin{enumerate}[(i)]
		\item Cities absorb population growth; 
		\item migration from rural areas;
		\item rural areas become reclassified as urban. 
	\end{enumerate}	

\begin{figure}[h]
	\centering
	\includegraphics[width=0.75\textwidth]{./img/world_total_urban.pdf} 
	\caption{Global urbanisation trends according to \citet{AREAS2012}.} \label{fig:urban_trends}
\end{figure}

	Urbanisation trends exhibit two notable features. Firstly, the majority of growth is concentrated in the developing world, with Africa and Asia together accounting for 86\% of the increase in urban dwellers by 2050 \citep{York2011}. Secondly, despite much excitement about `megacities', urban growth is mainly in smaller cities (below 500,000 people) in the foreseeable future). Cities of more than 10 million residents will accommodate less than 10\% of the urban population \citep{Cohen2006}.

These predictions aren't just the view of a small number, but widely accepted opinion. Thus even \citet{Cohen2004} who takes a skeptical view of urbanisation projections (pointing out that historical forecasts can be overestimates by more than 75\%; and that there is no universally consistent definition of `urban'\footnote{See for example \citet{DepartmentforCommunitiesandLocalGovernment2006}.}) concludes: \emph{``Nevertheless, despite all the problems of error and inaccuracy and the long-standing definitional problems that have never been overcome, it is clear that the world is still in the midst of a sweeping and profound urban transformation that is literally changing the face of the planet.''}

	\item The mechanism of urbanisation outlined by \citet{WorldBank2008a} shows how and why urbanisation occurs, relating the process to the economic \emph{importance} of urban areas. Urban centres arise as people migrate from the countryside in search of employment and better access to services (such as education and healthcare). As cities grow, they begin to specialise with firms taking advantages of economies of scale. Cities become `agglomeration economies' when multiple types of firms and industries occupy the same urban space, facilitating knowledge spillovers, and specialisation such that firms benefit from each other's presence. At the same time, transport (and hence trade) costs reduce. Therefore scale economies and agglomeration economies act to increase trade, and hence the prosperity of an area, attracting more migrants to the area. Thus a virtuous circle arises whereby city size and economic growth feed each other (see Figure~\ref{fig:urbanCycle}). This mechanism explains why it is estimated that 80\% of global GDP is generated in urban areas \citep{AREAS2012}.

\begin{figure}[h]
	\centering
	\input{./img/urbanCycle}
	\caption{Virtuous circle of urban growth and economic prosperity.} \label{fig:urbanCycle}
\end{figure}
Growing cities in developing countries aren't without their problems, for example, the widespread presence of slums. The World Bank describes these as `growing pains', and notes that they have historically existed in cities like London, but are now no longer to be seen. It is therefore confirmed that urbanisation is a clear long term route to economic prosperity. It is encouraging that the urbanisation patterns seen historically in OEDC countries are being replicated in the developing world \citep{WorldBank2008a}. 

	\item It follows therefore that urbanisation is \emph{inevitable} due to the economic prosperity it brings to the developing world. Thus it is troubling to learn that \emph{``\ldots 72\% of developing countries have adopted policies designed to stem the tide of migration to their cities.''} \citep{Donald2012}.
\end{enumerate}


\subsection{Urbanisation poses challenges} \label{sec:urban_challenges}
Having considered urbanisation trends and the associated economic benefits, two primary challenges of urbanisation are now discussed---those of \emph{environmental strain} and \emph{economic stability}.

\subsubsection*{Environmental strain} 
This section will consider environmental effects of urbanisation in general, before a more detailed discussion regarding the effects relating to the energy, water and waste sectors in particular and how they interact with each other. The policy and decision making environment which this creates will briefly be described and then London will be used as a case study to demonstrate the environmental challenges it faces, and their proposed solutions.

In general, urban areas both experience an internal environmental strain and exert an environmental strain externally. Examples of internal strain include land-use impacts, resource scarcity, poor water quality, emissions, the heat-island effect and eutrophication, much of which can be related to resource overconsumption \citep{Cao2011}. A society starts to overconsume resources when the area which serves it (the hinterland) is unable to provide enough food, water and other materials to meet the needs of its inhabitants. An increasing population requires a larger quantity of resources to meet its needs. To extract and manage these resources itself requires a growing population inducing a self-feeding cycle of population  growth and resource management. At some point, a `labour limit' is reached where the land can no longer sustain the workforce. Such resource scarcity can be true of both agricultural based societies as well as industrial based societies \citep{Haberl2001a, Haberl2001b, Gr2003}. In seeking to overcome these internal pressures on resources, cities look beyond the immediate hinterlands, making use of the transport system to become dependent on resource imports from global markets. Thus the environmental impact that cities are responsible for spreads such that it becomes an external, global problem \citep{Agudelo-Vera2011}. In summary, the internal environmental strain due to resource over-consumption leads to external impacts on a global scale.

The \emph{energy} sector evidences the global environmental impacts of urbanisation by the fact that whilst currently home to just over 50\% of the global population, urban areas are responsible for around 60\% of primary energy use and 71\% of the global energy related greenhouse gas emissions \citep{IEA2008}. There are multiple causes of this including the lifecycle heating and power demands, and urban sprawl leading to increases in transport use due to longer and more frequent journeys \citet{Grubler2009}.

In the \emph{water} sector, urbanisation is leading to `water stress' as populations grow and there is a demand for a higher quality of life. A case study is provided by \citet{Kennedy2008} who illustrates the problem of urban water sustainability by considering the relationship between urbanisation and groundwater. At the early stages of urban development, shallow wells are sufficient to supply a population, and wastewater can discharge to a watercourse. As the population grows and water extraction rates rise, the water table falls requiring the digging of deeper wells. Shallow groundwater can also become contaminated by wastewater discharge. Thus water must be sourced (at significant expense) from outside the city, changing the city's groundwater characteristics such that the water table can rise above its original level and even cause flooding. In summary then, water security issues in urban areas are causing financial, health and environmental concerns.
%Associated with the increasing demand is the need to treat the water and prevent contamination. This demand for water in high quantitiies comes with a financial burden \citep{Daigger2009}. 

\emph{Waste management} also presents urban areas with a number of issues. These include groundwater contamination, public opposition to waste disposal near residential areas \citep{Li2006}, as well as growing waste generation rates and decreasing disposal capacity \citep{Lu2009}. For example, \citet{Xydis2012} have described the waste management capacity limitation in Greece as having reached a `critical point'. The decomposition of waste in landfill sites contributes to global warming through the production of methane (in addition to the emissions from the transport and management of waste \citep{ICE2011}). The health risks posed by open sites in developing countries are also of concern \cite{Durand2013}.

As well as presenting challenges individually, the energy, water and waste sectors interact with each other, such that they feed each other, exacerbating the environmental challenge of resource management. 
\begin{itemize}
	\item \emph{Water and energy.} Both water and energy demands are exacerbated by population growth and economic growth (which induces a desire for improved living standards). The increase in water demands necessitates an increase in energy demands (for treatment and distribution). Furthermore, as climate change limits the availability of water in many places, energy consumption will need to increase for extracting water for human use (through desalination and other energy intensive treatment methods). It is estimated that approximately 35\% of the energy used by municipalities is for water supply and wastewater treatment \citep{El2005}. The increase in energy consumption will then exacerbate climate change invoking a self-reinforcing challenge \citep{Webber2011}. The water-energy link poses challenges in the other direction as water requirements for cooling (in power plants) increase to meet higher energy demands \citep{McMahon2011}. 
	\item \emph{Energy and waste.} The waste sector has a large energy consumption corresponding to transport and management processes \citep{ICE2011}.
	\item \emph{Water and waste.} The increase in water leads to increased wastewater generation which needs to be managed through either treatment or disposal \citep{McMahon2011}. 
\end{itemize}

Given the environmental challenges, it is no surprise that there is an increasing emphasis on developing policies and decisions to tackle the issues presented by urbanisation. For example, \citet{Agudelo-Vera2011} argues that the objective of urban planning needs to move on from simply solving the problems of accommodating larger populations and meeting transport needs, to include sustainable development. The authors contend that resource management is an essential part of sustainable development, and that resource management should move on from its historic aims of simply meeting demand. Thus, the discipline of urban planning should aim to end the pattern of over consumption and excessive waste production which the global ecosystem cannot carry. This approach is manifested in policies such as the European Union target to provide 20\% of final energy from renewable sources by 2020.

To bring together the interrelated resource management challenges and policies, the city of London is considered, because it is a good example of a city experiencing resource management problems and thus adopting policies and actions to meet the challenges.
\begin{itemize}
	\item In the \emph{energy} sector, London has a comprehensive range of policies concerning hybrid-fuelled transport, the retrofitting of housing and the decentralisation of energy supply in order to reduce energy-related emissions \citep{Strategy2011}. 
	\item In the \emph{water} sector, London is under large stress, suffering from an old network which leaks about 25\% of the water which enters it. When this is coupled with increasing demand (as population grows, with a tendency to live in houses of fewer people) and the reduced availability of water due to climate change, then it is obvious that current consumption (already above the national average) is unsustainable. Thus policies are being implemented to capture rainwater and wastewater as a resource; change consumer behaviour and payment patterns; and make buildings more efficient \citep{Nickson2011}. 
	\item In the municipal \emph{waste} sector, 49\% currently ends up in landfill, some of this outside London. Policies are being implemented to build new waste management infrastructure, increase recycling and use waste incineration as a method of energy production, for example with the SELCHP\footnote{South East London Combined Heat and Power.} incinerator \citep{Zabala2011}.
\end{itemize}
%\citep{Chen2006} Link to resource use, environment, policy and technology options.

%\citep{Batt2010} to summarise.
In summary, this study is in agreement with \citet{Newman1999} that the sustainability of cities needs to be improved through better resource management. It must be remembered however, that a city isn't simply a resource processing machine, but it is the home to many people, and the place where they seek economic opportunity.  Therefore, there is a particular challenge to better manage resources without impacting upon the `livability' of cities. %Another conclusion that is innevitable from this survey of challanges is that resource efficiency alone isn't enough. The world is already struggling to sustain the global population (for example, at current trends, the entire global biomass would be required to meet energy needs by 2050). Thus, a `per capita' decrease in consumption in a growing population will not be sufficient to meet the challenge. Moreover, more efficient use of resources may cause people to seek a higher standard of living---the so called `take-back' effect. There must be dematerialisation in absolute terms \citep{Winiwarter2011}.

\subsubsection*{Economic stability}
A second challenge is presented to the economic sustainability of cities. The argument here is based on the work of \citet{Bettencourt2007} who seek a quantitative understanding of \emph{``human social organization and dynamics in cities''}. As entities that both consume and produce resources, they are analogous to living organisms and ecosystems, whose properties (such as body size or growth rate) scale in a `self-similar' manner with mass (obeying \emph{Zipf's power law}). In the case of a city, indicators, $Y$ (such as GDP or energy consumption) scale with population, $N$ as:
\begin{align} \label{eq:urban_scale}
	Y(t)=Y_0N(t)^{\beta}
\end{align}
where $t$ is time, $Y_0$ is a normalisation constant and $\beta$ is a constant which corresponds to the indicator under consideration. It is observed that $\beta<1$ for indicators related to materials or infrastructure (such as the length of electrical cables), thus exhibiting economies of scale with population growth. However, for `innovation driven' indicators such as income, employment in R\&D and patents registered, $\beta>1$, thus the pace of life increases with respect to these indicators as population grows. %Thus material economies of scale is in tension with the super-linear scaling of innovation characteristics. 

If the growth of city is constrained by the availability of resources such that $R$ resources are required per unit time to maintain a member of the existing population, and $E$ extra resources are required to accommodate a new member, urban growth can be described as:
\begin{align} \label{eq:urban_growth}
	Y(t)=RN(t)+E\frac{dN(t)}{dt}
\end{align}
Equating Equations~\eqref{eq:urban_scale} and~\eqref{eq:urban_growth}, and rearranging, urban growth is modelled as:
\begin{align*}
	\frac{dN(t)}{dt}=\left(\frac{Y_0}{E}\right)N(t)^{\beta}-\left(\frac{R}{E}\right)N(t),
\end{align*}
which has the solution:
\begin{align*}
	N(t)=\left\{\frac{Y_0}{R}+\left(N^{1-\beta}(0)-\frac{Y_0}{R}\right)\exp\left[-\frac{R}{E}(1-\beta)t\right]\right\}^{\frac{1}{1-\beta}}
\end{align*}
This solution exhibits different behaviours, according to the situation:
\begin{enumerate}
	\item \emph{Scale economy driven growth.} As mentioned previously, in this case $\beta<1$, resulting in sigmoidal growth, such that the population will tend to a finite value.
	\item \label{itm:agglom} \emph{Agglomeration economy driven growth.} In this case, knowledge and innovation are the growth drivers, such that $\beta>1$.
	\begin{enumerate}
		\item \emph{Resources aren't at their limit.} $N$ will grow with $t$ at a greater-than-exponential rate, theoretically reaching an infinite population in a finite time, $t_c$:
\begin{align*}
	t_c&=-\frac{E}{(\beta-1)R}\ln\left[1-\frac{R}{Y_0}N^{1-\beta}(0)\right] \\
	&\approx \left[\frac{E}{(\beta-1)R}\right]\frac{1}{N^{\beta-1}(0)}.
\end{align*}
		\item \label{itm:collapse} \emph{Resources become limited.} However, since resources are limited, then there will come a time when they run out and stagnation and collapse occurs. 
	\end{enumerate}
\end{enumerate}

\begin{figure}[h!]
	\centering
	\includegraphics[width=0.75\textwidth]{./img/bettencourt_growthPlots.pdf} 
	\caption{Urban growth regimes taken from \citet{Bettencourt2007}. (a) Shows sublinear growth. (b) Shows exponential growth (linear). (c) Shows superlinear scaling before resources become limited. (d) Shows the economic collapse in the superlinear regime when resources become limited.} \label{fig:bettencourt_growth}
\end{figure}
These regimes are plotted in Figure~\ref{fig:bettencourt_growth}. To avoid the economic collapse of case \eqref{itm:collapse}, a transformation needs to be undergone which resets the initial conditions $Y_0$ and $N(0)$, to begin a new cycle of greater-than-exponential growth. However, $t_c$ for each cycle becomes shorter and shorter requiring innovation to take place at increasingly faster paces (see Figure~\ref{fig:bettencourt_innovation}). Thus, efficient resource management is a key challenge in urban design, because this effectively increases the value of $R$ and hence buys innovation time to push the stagnation point into the future.

\begin{figure}[h!]
	\centering
	\includegraphics[width=0.75\textwidth]{./img/bettencourt_innovation.pdf} 
	\caption{Representation of the innovation cycles necessary to postpone economic collapse taken from \citet{Bettencourt2007}. Note how the critical time $t_c$ reduces for each cycle.} \label{fig:bettencourt_innovation}
\end{figure}
% PLOT GRAPHS AT A LATER DATE

\subsubsection*{Summary}
In conclusion, there are two major challenges that urban planning must face. Firstly that of using resources more efficiently in order to minimise environmental strain experienced in and caused by cities. Secondly, in economies where innovation is the primary driver of growth, the economic collapse that can occur as cities outgrow their resource feed needs to be tackled by more efficient resource management in order to buy innovation time. In both cases, the root cause of the problems is an overconsumption of resources, and thus the solution is found in consuming fewer resources overall.

\clearpage
\subsection{Urban areas present opportunities}
\label{sec:urbanOpps}
Having seen how a more efficient use of resources is the route to meeting the problems caused by urbanisation, this section explores how urban areas are uniquely home to great opportunities to achieve this aim, chiefly due to the density of urban areas, and the resulting co-location of infrastructures. Thus, \emph{``urban planning in the 21\textsuperscript{st} Century should evolve towards an ecologically-oriented macro-architecture fully integrating the design and location of energy- and material-efficient buildings and urban infrastructure with overall spatial planning further to minimize material throughput.''} \citep{Rees1999}. 

The dense nature of cities is such that synergies between urban infrastructures can be identified and exploited. Thus a node in a resource management network can have multiple functionalities and the waste products of one process can become the inputs to another. Examples include waste-to-energy (WtE)---also known as energy-from-waste (EfW); biogas generation from the anaerobic digestion of sewage sludge; hydroelectric power generation; and the treatment and redistribution of wastewater \citep{Kharrazi2012}. This is what \citet{Leduc2010} refers to as the `urban harvest' approach, which takes advantage of connectivity and proximity in cities to maximise the reuse of water, waste and other materials within an urban system, and the cascading of energy through multiple uses in order to `close' cycles. Such `circularisation' of resources is increasingly playing a part in sustainable urban design \citep{Meijer2011}. The rest of this section goes on to outline some of the specific resource interactions which exist in urban environments, considering the energy and water sectors in turn. From this starting point, the resource integration opportunities possible at the `energy-water nexus' and the `energy-waste nexus' can be outlined.
% General
%The concepts that will be advocated include:
%- systems perspective
%- integration
%- decentralisation
%- closing the loop
%- cascading
%- synergies
%\citep{Cai2011} Systematic and intensive municiple infrastructure e.g. integrated water systems
%\citep{Clift2000} Systems approach

% Energy
%% Point out that there are inherent advantages to DER: price, reduces emissions, developing countires, fuel prices, low carbon construction
%% Specific resource integration advantages: local management, and local resources.
In the \emph{energy} sector, two key concepts are being advocated, those of `distributed' or `decentralised' provision and the notion of `energy cascading'. 
\begin{itemize} 
	\item \emph{Decentralisation} is where energy is generated closer to the point of use than traditional centralised generation, bringing multiple benefits. For example, in electricity networks, power losses are reduced (and hence emissions) along with electricity costs \citep{Fleten2007}. What makes distributed generation of specific interest to this study is the fact that these systems incorporate a \emph{``wide range of technologies including combined heat and power plant (CHP), photovoltaic systems (PV), small wind turbines and other systems using renewable energy sources (e.g. biogas digesters)''} \citep[p. 1001]{Ren2010} with the intention that local resources can be used as inputs (biomass, for example). Both the positioning of the systems in urban centres (at multiple locations) and the use of local resources facilitates their utility in taking advantage of resource intersections, infrastructure synergies and systems integration . Models already confirm the feasibility and environmental and financial benefits of integrating distributed systems and centralised systems, together with district heating networks in order to meet the energy demands of a city \citep{Weber2011}. 
	\item \emph{Energy cascading} is another (yet related) move in energy systems. An example is CHP, whereby waste heat in electricity production is used in heating systems \citep{Grubler2009}. In energy markets, energy cascading will manifest as companies provide `integrated energy services' (selling both heat and electricity to the consumer) \citep{Sugihara2004}. Biomass is another resource that can similarly be `cascaded' in ``energetic recycling" using residues and wastes in energy provision, rather than appropriating more natural resources for the same purpose \citep{Haberl2001a}.
\end{itemize}

The \emph{water} sector can also take advantage of both decentralisation and cascading in order to achieve integrated water management. \citet[p. 809]{Daigger2009} sees the answer to the problems of water stress in a toolkit of \emph{``closed-loop urban water and resource management systems''}, which includes: rainwater harvesting, conservation, reclamation and reuse, energy management, nutrient and source separation managed in centralised and decentralised forms, in order to move away from the linear \emph{``take, make, waste''} approach. For example, roof collectors and permeable pavements are decentralised methods of capturing rainwater; and houses can contain systems to separate kitchen water and grey water locally, for separate treatments and uses. Thus decentralisation occurs in the local collection and management (and even treatment) of water (even at the household scale), and cascading occurs as wastewater from one purpose is used in another which requires less stringent quality standards. This requires `source separation' to separate different streams of water for various uses---for example a potable/non-potable division means that water for use in toilet flushing requires less treatment than drinking water, resulting in the provision of additional water supplies to meet needs.
%%\citep{Kierstead2012b} Argues for integration (4.3)

% Water-energy-wastewater nexus
Having considered the shape of the energy and water sectors in isolation, the possible cross-sectoral interactions and intersections the urban environment gives rise to are now considered, starting with the \emph{energy-water nexus}\footnote{This discussion will include the place of wastewaster, thus it could be considered relevant to the waste-water nexus, which won't be considered separately at this point}. An extensive survey of both the energetic needs of water management and the opportunities for energy generation by water provided by \citet{McMahon2011} concludes there is a high level of interdependence between water and electricity which will increase in the future as global energy consumption increases. Resource integration opportunities arising out of these interdependencies are now discussed.
\begin{itemize}
	\item \emph{Closed-loop cooling} is advocated by \citet{McMahon2011} at the centralised level. This minimises the extensive water use required for cooling in electricity production. Here, water is used multiple times in cooling before discharge. This is an example of cascading.
	\item \emph{Anaerobic digestion} of wastewater is an option (at the centralised level), in order to convert the organic content of wastewater to biogas. At the decentralised level, multiple pipe networks can separate kitchen and toilet water (which has high organic content) for use in energy production in this way.
	\item \emph{Reclamation} and subsequent treatment of water reduces its energy requirements (compared to the treatment and distribution in ordinary water supply). \citet{Lundin2000} gives evidence that recycling wastewater makes emissions savings.
	\item \emph{Nutrient recovery} from wastewater saves energy through displacing it from  energy-intensive processes such as fertiliser production. 
\end{itemize}
In summary, the co-location of water and energy infrastructure means synergies can be exploited in an `integrated water system' such that both net water consumption and energy performance improves \citep{Makropoulos2008}.
%% INCLUDE FIG 5 and 6. 
%%challenge of bioenergy increase in water use. (McMahon).
%%\citep{Lim2010} Heirarchical water allocation, integration of infrastructure, reducing electricity consumption from pumping etc.
%% footnote on how wastewater is classified is required.
%%\citep{Makropulos2008} Tool for IWM based on an holistic view. Tradeoffs
%%\citep{Nickson2011} Rainwater, waste-water (for energy) and reclaimed water (reduces pumping).

The \emph{waste-energy nexus} is also receiving a lot of attention as another opportunity for urban resource integration, with reuse, recycling and recovery are becoming more popular \citep{Geng2010}. 
\begin{itemize} 
	\item \emph{Incineration} is popular method of waste management because it both saves space and can generate power and heat through WtE. (Although, very often, energy generated isn't reclaimed).  
	\item \emph{`Industrial symbiosis'} is advocated whereby firms use each others' waste outputs as raw material inputs in other processes (e.g. production and manufacturing as well as WtE applications), in line with the `proximity principle' observed in Japan, whereby waste is managed close to source. %For example, the OWARE model of \citep{Eriksson2002} takes a systems perspective on all the possible sources and uses of waste in a system (such as a city) by combining various sub-models which consider the life-cycle costs associated with incinceration, landfill, recyling , anaerobic digestion and so on.
	\item \emph{Anaerobic co-digestion} of sewage sludge and food waste is another waste-energy intersection (also intersecting with the water sector) which is discussed by \citet{Iacovidou2012}. This is a source of renewable energy, reducing landfill. Furthermore waste products can be used in fertiliser production).
\end{itemize}
A simple model to optimise the flows in an energy-waste network is proposed by \citet{Kharrazi2012} as an example of an integrated resource network. Nodes represent the producers and consumers of waste and electricity, and links between the nodes are the flows of those resources. A real world application of WtE networks is seen in Greece bringing two-fold benefits: mitigating against landfill shortages and producing renewable energy \citep{Xydis2012}.

\subsubsection*{Summary}
There are many opportunities in the energy, water and waste sector for more efficient resource management through systems integration within urban areas. This is made necessary in order to meet the challenges facing urban areas which arise from inefficient resource management.
%An underlying theoretical framework is required which can help us measure the scale of the challenges presented; the opportunities on offer; and the success of the models. The 'urban metabolism' concept will serve this purpose.
%
%Many studies suggest integrating infrastructures within their own field. This project is just an extension of that concept.
%
%The definition of integration (literature may be different from this study).
%The term `integration' can have multiple meanings
%Furthermore, there is little research in this field (despite many calls for it).  
\subsection{Urban metabolism---the theoretical framework}
It has been argued that cities are overconsuming resources, yet that they are also home to the opportunities for more efficient resource management. An underlying theoretical concept is required which can quantify the efficiency of urban resource use. To this end, the `urban metabolism' (UM) concept is introduced. Thus, the discussion surrounding the linearisation and circularisation of resources is actually a comment on the nature of the metabolism of an environment. UM is useful in three ways:
\begin{enumerate}
	\item UM studies provide data which \emph{motivates} the development of resource integration models. 
	\item It quantitatively demonstrates the \emph{means} by which urban resource management can be improved: namely, a move away from resource `linearisation' towards `circularisation'. 
	\item It provides a way of \emph{measuring} the success of the models. 
\end{enumerate}
In this section, UM will be defined to show how it indicates and quantifies the \emph{means} of resource integration. Then UM studies will be cited demonstrating such means by which cities can meet challenges and opportunities.

Conducting a UM study can be used to understand the processes taking place within it. Such studies imagine the city as an organism (or an ecosystem) and aim to quantify the fluxes of water, energy, waste and other materials into and out of urban populations in order to find the \emph{``sum total of the technical and socioeconomic processes that occur in cities, resulting in growth, production of energy and elimination of waste''} \citep[p. 44]{Kennedy2008}. It is a broad discipline, and as such individual studies vary in their scope, about the inclusion of particular resources, and whether the study is related to socioeconomic or political concerns (as in \citet{Hobbes2007}). The key features of a UM study are now described.
\begin{itemize}
	\item The basic \emph{methdology} in a UM study is a `material flow analysis' (MFA) where resource flows in and out of areas are measured to see where they accumulate \citep{Barles2009}. Inputs included `domestic extraction' (DE) or `imports' whilst outpus are `exports', `deliberate disposal' (DD) or `wastes and emissions' (WE). 
	\item The basic \emph{results} of an UM study are `indicators' given by operations on these variables. These can be used to compare the metabolism of different locations (or the same location at different times) to assess the relative sustainability of an area\footnote{For example, $\mbox{`material intensity'}=\mbox{Direct material input}/\mbox{GDP}$. This value reduces as a society dematerialises.} \citep{Hobbes2005}. \citet{Haberl2001a, Haberl2001b} show that the indicator `human appropriation of net primary production' (HANNP), `domestic consumption' and `total energy input' divides societies into three categories: hunter-gather, agricultural and industrial. Each of these exhibits qualitatively different metabolic profiles. Hunter-gather communities have no struggle meeting their needs from their environment, however agricultural communities will reach a `labour' limit where the environment will not be able to meet the needs of those working on it (see Section~\ref{sec:urban_challenges}), because their energetic needs (including that embodied in food and materials) simply cannot be met by the `biological productivity' of the area---thus even agricultural regimes can be unsustainable. 
\end{itemize}
Thus, UM is an accounting tool which offers a way to quantify the sustainability challenges presented by the overconsumption of resources. Hence it can be a useful consideration in sustainable urban design. \citet{Barles2010} makes the case for its importance in sustainable urban development issues, noting how research has moved on from viewing cities as `parasites', feeding off the environment at its expense. Instead, the UM approach has allowed the consideration of optimisation, making use of possible industrial symbioses, thus moving from linear to circular resource management in cities. Some specific results and features of UM studies which are pertinent to resource management efficiency are now discussed.
\begin{itemize}
	\item \emph{In general, there is an increasing per capita metabolism.} \citet{Kennedy2008} demonstrates the power of the UM concept in reviewing the studies of eight regions around the world since 1965. The authors conclude that in general, there is an increasing per capita metabolism for water, wastewater, energy and materials (therefore cities are assimilating mroe of these resources per person). The largest components in UM studies tend to be energy, water and waste, thus justifying the approach of this study in targetting these resources for resource integration opportunities. London is a good example of this result \citep{BFF2002}.
	\item \emph{Improvements in metabolism are observed in some areas.} Despite the global trends, some places (such as Toronto) have seen per capita reductions in energy and water use. This has been achieved through increased efficiency. Similarly, multiple cities have increased recycling rates and hence reduced per capita waste consumption \citep{Kennedy2008}. 
	\item \emph{High metabolism and economic indicators are linked}. A particular challenge noted in the literature is the coupling between direct material consumption (DMC), direct material inputs (DMI) and GDP, for example, in the case of Singapore \citep{Schulz2007a}. Thus \citet{Kowalski1997} argues the challenge that sustainable urban design faces is the need to de-link prosperity and quality of life from metabolism. This can be achieved as technologies increase in efficiency faster than the economic growth of a society (as well as bringing about changes in cultural expectations and behaviours). The way that such efficiency improvements can be achieved at the urban scale (as opposed to the individual technology scale) is through the circularisation of urban resource flows (rather than the current linear approach)---which ultimately reduces inputs, as argued by \citet{Newman1999} who advocates the `extended metabolism model'. This approach wants to recognise that a city is more than a mechanism for the physical processing of resources, but also the centre of human opportunity. 
\end{itemize} 
The results of these UM studies give the \emph{means} to achieving improved resource efficiency by identifying specific areas for circularisation and dematerialisation. For example, a multi-scale UM study of Paris and the surrounding region in \citet{Barles2009} shows the city's dependence on a large area for its waste management. Thus, actions can be targeted in that sector to improve Paris's sustainability. Having an appreciation for the utility of the UM concept in assessing the resource efficiency of an area, it can be seen why \citet{Kennedy2011} advocates UM studies in design, using MFAs to produce \emph{``more ecologically sensitive designs''} of a city through infrastructure integration, closing loops to minimise overall inputs and outputs of resources. In conclusion, it can be said that the UM concept is key to improving resource management in order to bring sustainable design considerations to urban planning \citep{Agudelo-Vera2011}.
%% definitions
%% challenges
% \citep{Chen2012} Measure of metabolism given by network environ theory. Indicators for linkage and synergism.
%% Opportunities
%% Policy and planning
% \citep{Hobbes2007} social context to relate UM to actors, decisions. Explanatory approach to MFA.

\subsection{Aims and scope of the study} \label{sec:aims}
Having now outlined the context and motivation for the study, a research question is posed. This gives the study a clear focus, and provides a benchmark for assessing the success of the final outcome of the study.
\input{./tex/aims.tex}

%\subsection{Report structure}
%Having presented the background to the study to provide motivation for the research question posed, the report continues by reviewing existing methods of optimising the provision of energy, water and waste in Section~\ref{sec:methods}. In Section~\ref{sec:models}, prototype models at their initial stage are presented. The report conludes in Section~\ref{sec:conc} with preliminary results, and an outline of the work to be undertaken in the rest of the project.
%
%In summary, it has been argued cities need to focus on resource integration in order to reduce damage to the environment and economic risk. Currently, there are no models which consider the optimal management of multiple urban resources.

\subsection*{Summary}
In conclusion, there is strong case to pursue research into a model that optimises the provision of urban energy, water and waste resources. This is because urbanisation has socioeconomic advantages, such that it should not be resisted, despite the environmental and economic challenges brought by resource overconsumption. Rather, urban centres provide an opportunity for more efficient resource management by exploiting the proximity and co-location of infrastructure. The urban metabolism concept can be used to quantify the resource efficiency of a city, hence providing an underlying theoretical framework to the project.%he challenges of resource management. On top of this there is a research-gap in the field of integrated resource provision.

%\input{./tex/methods.tex}
%\input{./tex/models.tex}
%\input{./tex/conc.tex}
%\appendix
%\usection{Tat Hamlet resource flow model code} \label{sec:code_tat}
%\subsection{Optimisation script}
%\lstinputlisting{./code/tat_model.py}
%\clearpage
%\subsection{Data script}
%\lstinputlisting{./code/tat_data.dat}
%\section{Water quality model}
%\subsection{Optimisation script}
%\lstinputlisting{./code/water_model.py}
%\clearpage
%\subsection{Data script}
%\lstinputlisting{./code/water_data.dat}
%\section{Gantt chart} \label{sec:Gantt}
%\includepdf[pages={-}, landscape=true]{Gantt.pdf}

%\bibliographystyle{kbib}
%\addcontentsline{toc}{section}{References}
%\bibliography{9mnth}

\end{document}
